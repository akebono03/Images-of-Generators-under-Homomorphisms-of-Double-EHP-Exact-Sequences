\documentclass{article}
\usepackage[a4paper,margin=2cm,landscape]{geometry} % 横向き設定
\usepackage{pb-diagram}
\usepackage{amsmath}
\usepackage{tikz-cd}
\usetikzlibrary{arrows.meta}
\tikzcdset{
every arrow/.append style={thick},
arrow style = tikz,            % TikZ 標準の矢印を使う
%diagrams = {>={Stealth[round]}} % 矢尻を Stealth(丸み付き)に
diagrams = {>={Latex[round]}}
}
\setlength{\dgARROWLENGTH}{1.4em}

\begin{document}
\(
\begin{tikzcd}[nodes={font=\Huge},row sep=0.2cm,column sep=0.1cm]
n= & & 3 & & 5 & & 7 & & 9 & & 11 & & 13 & & 15 & & 17 &  \\
 & & & w \\
k=29 & & & & \bullet \arrow[lu] & = & \bullet & = & \bullet & = & \bullet & = & \bullet & = & \bullet & = & & \{\alpha_1\beta_2\} \\
& & & ab^2 & & & & & & \\
k=30 & & & & \bullet \arrow[lu] & = & \bullet & = & \bullet & = & \bullet & = & \bullet & = & \bullet & = & \bullet & = \{\beta_1^3\} \\
& b^3 & & & & & & & & AB & & b \\
k=33 & & \bullet \arrow[lu] & = & \bullet & = & \bullet & = & \bullet & = & \bullet & \rightarrow & \circ \arrow[lu] & \rightarrow & \bullet \\
 & & & & & & & & & & \bullet & & {} & & {} &  \\
 & & & & & & & & & & & B & & a_2 \arrow[lu] & & a \arrow[lu]
\end{tikzcd}
\)
\end{document}


