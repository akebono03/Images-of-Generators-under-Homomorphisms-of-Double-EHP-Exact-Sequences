\documentclass{article}
\usepackage{pb-diagram}
\setlength{\dgARROWLENGTH}{1em}

\begin{document}
\[
  \begin{diagram}
    \node[2]{Q^1(\alpha_1)} \\
    \node{} 
      \node[2]{\pi_{2q+1}(S^3)}
      \arrow{nw,t}{H}
      \arrow[2]{e,t}{E^2}
      \node[2]{\pi_{2q+3}(S^5)}
      \arrow[2]{e,t}{E^2}
      \node[2]{\pi_{2q+5}(S^7)}
      \arrow[2]{e,t}{E^2}
      \node[2]{\cdots ,} \\
    \node[2]{\overline{Q}^1(\alpha_1)}
    \node[2]{Q^2(\iota)}
    \arrow{nw,t}{P} \\
    \node{}
      \node[2]{\pi_{2q+2}(S^3)}
      \arrow{nw,t}{H}
      \arrow[2]{e,t}{E^2}
      \node[2]{\pi_{2q+4}(S^5)}
	\arrow{nw,t}{H}
      \arrow[2]{e,t}{E^2}
      \node[2]{\pi_{2q+6}(S^7)}
      \arrow[2]{e,t}{E^2}
      \node[2]{\cdots ,} \\
  \end{diagram}
\]

\end{document}


