\documentclass{article}
\usepackage[a4paper,margin=2cm,landscape]{geometry} % 横向き設定
\usepackage{pb-diagram}
\setlength{\dgARROWLENGTH}{1em}

\begin{document}
\(
  \begin{diagram}
    \node[2]{\pi_{-1+k}(Q_2^1)}
    \node[2]{\pi_{1+k}(Q_2^3)}
    \node[2]{\pi_{3+k}(Q_2^5)} \\
    \node{} 
      \node[2]{\pi_{2+k}(S^3)}
      \arrow{nw,t}{H}
      \arrow[2]{e,t}{E^2}
      \node[2]{\pi_{4+k}(S^5)}
      \arrow{nw,t}{H}
      \arrow[2]{e,t}{E^2}
      \node[2]{\pi_{6+k}(S^7)}
      \arrow{nw,t}{H}
      \arrow{e,t}{E^2}
      \node{\cdots ,} \\
    \node[2]{\pi_{k}(Q_2^1)}
    \node[2]{\pi_{2+k}(Q_2^3)}
    \arrow{nw,t}{P} 
    \node[2]{\pi_{4+k}(Q_2^5)}
    \arrow{nw,t}{P} 
    \\
    \node{}
      \node[2]{\pi_{3+k}(S^3)}
      \arrow{nw,t}{H}
      \arrow[2]{e,t}{E^2}
      \node[2]{\pi_{5+k}(S^5)}
	\arrow{nw,t}{H}
      \arrow[2]{e,t}{E^2}
      \node[2]{\pi_{7+k}(S^7)}
	\arrow{nw,t}{H}
      \arrow{e,t}{E^2}
      \node{\cdots ,}
  \end{diagram}
\)

\end{document}


