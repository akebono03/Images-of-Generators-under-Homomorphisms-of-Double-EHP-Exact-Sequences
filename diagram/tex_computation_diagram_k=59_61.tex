\documentclass{article}
\usepackage[a4paper,margin=2cm,landscape]{geometry} % 横向き設定
\usepackage{pb-diagram}
\usepackage{amsmath}
\usepackage{tikz-cd}
\tikzcdset{
%every arrow/.append style={thick},
arrow style = tikz,            % TikZ 標準の矢印を使う
%diagrams = {>={Stealth[round]}} % 矢尻を Stealth(丸み付き)に
diagrams = {>={Latex[round]}}
}
\setlength{\dgARROWLENGTH}{1.0em}
%\usepackage[left=0cm]{geometry}
\pagestyle{empty}

\begin{document}
\(
\begin{tikzcd}[
nodes={font=\large,inner sep=0.2pt},
row sep=0.4cm,
column sep=0.2cm
%column sep=0.01em
]
n=& && 3 & & 5 & & 7 & & 9 & & 11 & & 13 & & 15 & & 17 & & 19 & & 21 & & 23 && 25 && 27 \\
&& b^3b_2 && && \\
k=59& && \bullet\arrow[lu] &=& \bullet &=& \bullet \\
&& U && u_2 && u_1 && ph\arrow[lu] && && && && && && && && \\
k=60& && \bullet\arrow[lu] && \bullet\arrow[lu] &\hookleftarrow& \bullet\arrow[lu] &&  \\
&& && U_2\arrow[lu] && U_1\arrow[lu] && w\arrow[lu] && && && && AB_2 && b_2 && && && && \\ 
k=61& && && && && && && && && \bullet\arrow[lu] & \hookleftarrow & \bullet\arrow[lu] &=& \bullet &=& \bullet &=& \bullet \\
&& && ab_2^2 && && B^2B_2 && && abb_2 && && && B_2\arrow[lu] && ab^2 && && && B\arrow[lu] \\
\end{tikzcd}
\)
\end{document}


