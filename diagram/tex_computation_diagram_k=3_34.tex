\documentclass{article}
%\usepackage[a4paper,margin=2cm,landscape]{geometry} % 横向き設定
\usepackage{pb-diagram}
\usepackage{amsmath}
\usepackage{tikz-cd}
\usetikzlibrary{arrows.meta}
\tikzcdset{
every arrow/.append style={thick},
arrow style = tikz,            % TikZ 標準の矢印を使う
%diagrams = {>={Stealth[round]}} % 矢尻を Stealth(丸み付き)に
diagrams = {>={Latex[round]}}
}
\setlength{\dgARROWLENGTH}{1.4em}
\usepackage[left=0cm]{geometry}
\pagestyle{empty}

\begin{document}
\(
\begin{tikzcd}[nodes={font=\huge,inner sep=0pt},row sep=0.2cm,column sep=0.1cm]
n= & & 3 & & 5 & & 7 & & 9 & & 11 & & 13 & & 15 & & 17 &  \\
k=3 && \bullet & = && && && && && && && \{\alpha_1\} \\
k=6 && \bullet \\
k=7 && \bullet & = & \bullet & = && && && && && && \{\alpha_2\} \\
k=10 && \bullet & \rightarrow & \circ & \rightarrow & \bullet & = && && && && && \{\beta_1\} \\
k=11 && \bullet & \rightarrow & \circ & = & \circ & = && && && && && \{\alpha'_3\} \\
k=13 && \bullet & = & \bullet &=& \bullet &=&& && && && && \{\alpha_1\beta_1\} \\
k=14 && \bullet && \bullet && \bullet \\
k=15 && \bullet &=& \bullet &=& \bullet &=& \bullet &= && && && && \{\alpha_4\} \\
k=16 && \bullet \\
k=17 && \bullet \\
k=18 && \bullet && \bullet && \bullet && \bullet \\
k=19 && \bullet &=& \bullet &=& \bullet &=& \bullet &=& \bullet &=&& && && \{\alpha_5\} \\
k=20 && && \bullet &=& \bullet &=& \bullet &=& \bullet &=&& && && \{\beta_1^2\} \\
k=21 && && \bullet && \bullet &=& \bullet \\
k=22 && \bullet & \rightarrow & \circ & \rightarrow & \circ & \rightarrow & \circ & \rightarrow & \bullet \\
k=23 && \bullet & \rightarrow & \circ & = & \circ & = & \circ & = & \circ & = & \circ & = && && \{\alpha'_6\} \\
 && \bullet &=& \bullet &=& \bullet &=& \bullet &=& \bullet &=& \bullet & = && && \{\alpha_1\beta_1^2\} \\
\end{tikzcd}
\)
\end{document}


