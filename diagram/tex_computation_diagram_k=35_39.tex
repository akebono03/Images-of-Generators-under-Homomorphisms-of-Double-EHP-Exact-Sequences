\documentclass{article}
%\usepackage[a4paper,margin=2cm,landscape]{geometry} % 横向き設定
\usepackage{pb-diagram}
\usepackage{amsmath}
\usepackage{tikz-cd}
\usetikzlibrary{arrows.meta}
\tikzcdset{
every arrow/.append style={thick},
arrow style = tikz,            % TikZ 標準の矢印を使う
%diagrams = {>={Stealth[round]}} % 矢尻を Stealth(丸み付き)に
diagrams = {>={Latex[round]}}
}
\setlength{\dgARROWLENGTH}{1.4em}
\usepackage[left=0cm]{geometry}
\pagestyle{empty}

\begin{document}
\(
\begin{tikzcd}[nodes={font=\Large},row sep=0.2cm,column sep=0.1cm]
n= & & 3 & & 5 & & 7 & & 9 & & 11 & & 13 & & 15 & & 17 && 19 && 21  \\
k=35 && \bullet &\rightarrow& \circ &\rightarrow& \triangleright &=& \triangleright &=& \triangleright &=& \triangleright &=& \triangleright  &=& \triangleright &=& \triangleright &=& & \{\alpha''_9\} \\
k=36 && && \bullet &=& \bullet &=& \bullet &=& \bullet &=& \bullet &=& \bullet &=& \bullet &=& \bullet &=&& \{\beta_1\beta_2\} \\
&& \bullet && && && \bullet &=& \bullet && \bullet \\
k=37 && \bullet && \bullet && \bullet &=& \bullet &=& \bullet &=& \bullet &=& \bullet &=& \bullet &=& \bullet &=&& \{\varepsilon'\} \\
k=38 && && && && \bullet && && && \bullet &=& \bullet &=& \bullet &=& \bullet & =\{\varepsilon_1\} \\
 && \bullet && \bullet && \bullet && \bullet && \bullet && \bullet && \bullet && \bullet && \bullet \\
k=39 && \bullet &=& \bullet &=& \bullet &=& \bullet &=& \bullet &=& \bullet &=& \bullet &=& \bullet  &=& \bullet &=& \bullet& = \{\alpha_{10}\} \\
 && && && && \bullet && && && \\
&& \bullet &=& \bullet &=& \bullet &=& \bullet &=& \bullet &=& \bullet &=& \bullet &=& \bullet  &=& \bullet &=& \bullet& = \{\alpha_{1}\beta_1\beta_2\} \\
\\
\end{tikzcd}
\)
\end{document}


