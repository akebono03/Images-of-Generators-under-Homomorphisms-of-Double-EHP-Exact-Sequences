\documentclass{article}
% \usepackage[a4paper,margin=2cm,landscape]{geometry} % 横向き設定
\usepackage{pb-diagram}
\usepackage{amsmath}
\usepackage{tikz-cd}
\usetikzlibrary{arrows.meta}
\tikzcdset{
every arrow/.append style={thick},
arrow style = tikz,            % TikZ 標準の矢印を使う
%diagrams = {>={Stealth[round]}} % 矢尻を Stealth(丸み付き)に
diagrams = {>={Latex[round]}}
}
\setlength{\dgARROWLENGTH}{1.4em}
\usepackage[left=0cm]{geometry}


\begin{document}
\(
\begin{tikzcd}[
nodes={font=\large},
row sep=0.3cm,
column sep=0.1cm,
column 16/.append style={nodes={align=left,anchor=west}}
]
n= & & 3 & & 5 & & 7 & & 9 & & 11 & & 13 & & 15 & & 17 & & 19 & & 21 & & 23 & \\
& & & ab_2 & & \\ 
k=36 & & & & \bullet \arrow[lu] & = & \bullet & = & \bullet & = & \bullet & = & \bullet & = & \bullet & = & \bullet & = & \bullet & = & & & & \{\beta_1\beta_2\} \\
& & & AB_2 & & b_2 \\
k=37 & & & & \bullet \arrow[lu] & \hookleftarrow & \bullet \arrow[lu] & = & \bullet & = & \bullet & = & \bullet & = & \bullet & = & \bullet & = & \bullet & = & & & & \{\varepsilon'\}  \\
& & & & & B_2 \arrow[lu] & & & & & & & & B \\
k=38 & & & & & & & & & & & & & & \bullet \arrow[lu] & = & \bullet & = & \bullet & = & \bullet & = & & \{\varepsilon_1\} \\
& bb_2  & & & & & & & & & & \\ 
k=39 & & \bullet \arrow[lu] & = & \bullet & = & \bullet & = & \bullet & = & \bullet & = & \bullet & = & \bullet & = & \bullet & = & \bullet & = & \bullet & = & & \{\alpha_1\beta_1\beta_2\} \\
& BB^2 & & & &  \\
k=40 & & \bullet \arrow[lu] & = & \bullet & = & \bullet & = & \bullet & = & \bullet & = & \bullet & = & \bullet & = & \bullet & = & \bullet & = & \bullet & = & & \{\beta_1^4\} \\
& & & & & & & & & ab^2 \\
k=42 & & & & & & & & & & \bullet \arrow[lu] & = & \bullet & = & \bullet & = & \bullet & = & \bullet & = & \bullet & = & \bullet & =\{\varepsilon_2\} \\
& & & & & w & & b^3 \\ 
k=45 & & & & & & \bullet \arrow[lu] & \rightarrow & \circ \arrow[lu] & = & \circ & = & \circ & = & \circ & = & \circ & = & \circ & = & \circ & = & \circ & = \{\varphi\} \\
n= & & 3 & & 5 & & 7 & & 9 & & 11 & & 13 & & 15 & & 17 & & 19 & & 21 & & 23 & \\
\end{tikzcd}
\)
\end{document}


