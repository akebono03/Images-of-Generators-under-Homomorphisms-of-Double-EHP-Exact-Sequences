\documentclass{article}
% \usepackage[a4paper,margin=2cm,landscape]{geometry} % 横向き設定
\usepackage{pb-diagram}
\usepackage{amsmath}
\usepackage{tikz-cd}
\usetikzlibrary{arrows.meta}
\tikzcdset{
every arrow/.append style={thick},
arrow style = tikz,            % TikZ 標準の矢印を使う
%diagrams = {>={Stealth[round]}} % 矢尻を Stealth(丸み付き)に
diagrams = {>={Latex[round]}}
}
\setlength{\dgARROWLENGTH}{1.4em}

\begin{document}
\(
\begin{tikzcd}[
nodes={font=\Large},
row sep=0.3cm,
column sep=0.1cm,
column 16/.append style={nodes={align=left,anchor=west}}
]
n= & & 3 & & 5 & & 7 & & 9 & & 11 & & 13 & & 15 & {} \\
 & & & a & & \\
k=10 & & & & \bullet \arrow[lu] & = & \bullet & = & & & & & & & &  \{\beta_1\} \\
k=11 & {} \\
k=12 & {} \\
& b & {} & {} & {} & {} \\
k=13 & {} & \bullet \arrow[lu] & = & \bullet & = & \bullet & = & {} & {} & {} & {} & {} & {} & {} & {} \{\alpha_1\beta_1\} \\
k=14 & {} \\
k=15 & {} \\
& ab & {} \\
k=16 & & \bullet \arrow[lu] \\
& AB & & b \arrow[lu] \\
k=17 & & \bullet \arrow[lu] \\
& & & B \arrow[lu] \\
k=18 & {} \\
k=19 & {} \\
\end{tikzcd}
\)
\end{document}


