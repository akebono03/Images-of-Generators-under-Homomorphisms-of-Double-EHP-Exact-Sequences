\documentclass{article}
%\usepackage[a4paper,margin=2cm,landscape]{geometry} % 横向き設定
\usepackage{pb-diagram}
\usepackage{amsmath}
\usepackage{tikz-cd}
\usetikzlibrary{arrows.meta}
\tikzcdset{
every arrow/.append style={thick},
arrow style = tikz,            % TikZ 標準の矢印を使う
%diagrams = {>={Stealth[round]}} % 矢尻を Stealth(丸み付き)に
diagrams = {>={Latex[round]}}
}
\setlength{\dgARROWLENGTH}{1.4em}
\usepackage[left=0cm]{geometry}
\pagestyle{empty}

\begin{document}
\(
\begin{tikzcd}[nodes={font=\large},row sep=0.2cm,column sep=0.1cm]
n= & & 3 & & 5 & & 7 & & 9 & & 11 & & 13 & & 15 & & 17 && 19 && 21 && 23 \\
k=40 && \bullet &=& \bullet &=& \bullet &=& \bullet &=& \bullet &=& \bullet &=& \bullet &=& \bullet &=& \bullet &=&&&& \{\beta_1^4\} \\
&&&& \bullet && \bullet && && && && \bullet &&  \\
k=41 &&&& \bullet && \bullet && && && && \bullet &&  \\
k=42 && \bullet && && && && \bullet &=& \bullet &=& \bullet &=& \bullet  &=& \bullet  &=& \bullet  &=& \bullet & =\{\varepsilon_2\} \\
 && \bullet && \bullet && \bullet && \bullet && \bullet && \bullet && \bullet && \bullet && \bullet && \bullet \\
k=43 && \bullet &=& \bullet &=& \bullet &=& \bullet &=& \bullet &=& \bullet &=& \bullet &=& \bullet  &=& \bullet &=& \bullet&=& \bullet & = \{\alpha_{11}\} \\
&& \bullet && && && \\
&& \bullet && && && && \bullet && \bullet &=& \bullet   \\
\\
\end{tikzcd}
\)
\end{document}


