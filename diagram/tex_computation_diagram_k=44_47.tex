\documentclass{article}
%\usepackage[a4paper,margin=2cm,landscape]{geometry} % 横向き設定
\usepackage{pb-diagram}
\usepackage{amsmath}
\usepackage{tikz-cd}
\usetikzlibrary{arrows.meta}
\tikzcdset{
every arrow/.append style={thick},
arrow style = tikz,            % TikZ 標準の矢印を使う
%diagrams = {>={Stealth[round]}} % 矢尻を Stealth(丸み付き)に
diagrams = {>={Latex[round]}}
}
\setlength{\dgARROWLENGTH}{1.4em}
\usepackage[left=0cm]{geometry}
\pagestyle{empty}

\begin{document}
\(
\begin{tikzcd}[
nodes={font=\Large,inner sep=0pt},
row sep=0.2cm,
column sep=0.1cm
%column sep=0.01em
]
n= && & 3 & & 5 & & 7 & & 9 & & 11 & & 13 & & 15 & & 17 && 19 && 21 && 23 && 25\\
k=44 &&& \bullet && && && \bullet &&  \\ \\
k=45 &&&&& && \bullet &\rightarrow& \circ &=& \circ &=& \circ &=& \circ &=& \circ &=& \circ &=& \circ &=& \circ &=& & \{\varphi\}  \\
& && && && && && && && && \bullet && \bullet &=& \bullet  \\ \\
k=46 &&& && \bullet &=& \bullet &=& \bullet &=& \bullet &=& \bullet  &=& \bullet  &=& \bullet  &=& \bullet &=& \bullet  &=& \bullet  &=& \bullet  & =\{\beta_1^2\beta_2\} \\
 &&& && && && \bullet &=& \bullet && \bullet && \\
&&& \bullet &\rightarrow& \circ &\rightarrow& \circ &\rightarrow& \circ &\rightarrow& \circ &\rightarrow& \circ &\rightarrow& \circ &\rightarrow& \circ &\rightarrow& \circ &\rightarrow& \circ &\rightarrow& \bullet \\ \\ 
k=47 &&& \bullet &\rightarrow& \circ &=& \circ &=& \circ &=& \circ &=& \circ &=& \circ &=& \circ &=& \circ &=& \circ &=& \circ &=& \circ& = \{\alpha'_{12}\} \\
&&& && \bullet && \bullet &=& \bullet && \\
&&& && \bullet &=& \bullet &=& \bullet &=& \bullet &=& \bullet  &=& \bullet  &=& \bullet  &=& \bullet &=& \bullet  &=& \bullet  &=& \bullet  & =\{\beta_1\varepsilon'\} \\
\\
\end{tikzcd}
\)
\end{document}


