\documentclass{article}
% \usepackage[a4paper,margin=2cm,landscape]{geometry} % 横向き設定
\usepackage{pb-diagram}
\usepackage{amsmath}
\usepackage{tikz-cd}
\usetikzlibrary{arrows.meta}
\tikzcdset{
every arrow/.append style={thick},
arrow style = tikz,            % TikZ 標準の矢印を使う
%diagrams = {>={Stealth[round]}} % 矢尻を Stealth(丸み付き)に
diagrams = {>={Latex[round]}}
}
\setlength{\dgARROWLENGTH}{1.4em}
\usepackage[left=0cm]{geometry}


\begin{document}
\(
\begin{tikzcd}[
nodes={font=\Large},
row sep=0.3cm,
column sep=0.1cm,
column 16/.append style={nodes={align=left,anchor=west}}
]
n= & & 3 & & 5 & & 7 & & 9 & & 11 & & 13 & & 15 & & 17 & & 19 & & 21 & & \\
& AE_2 && e_2 && e_1 && && AB_2 && b_2 \\ 
k=49 && \bullet \arrow[lu] && \bullet \arrow[lu] & \hookleftarrow & \bullet \arrow[lu] & = & \bullet & & \bullet \arrow[lu] & \hookleftarrow & \bullet \arrow[lu] & = & \bullet & = & \bullet & = & \bullet & && \\
& && E_2 \arrow[lu] && E_1 \arrow[lu] && && B^3 \arrow[lu] && B_2 \arrow[lu] && && && && B \arrow[lu] \\
\end{tikzcd}
\)
\end{document}


