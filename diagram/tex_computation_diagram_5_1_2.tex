\documentclass{article}
% \usepackage[a4paper,margin=2cm,landscape]{geometry} % 横向き設定
\usepackage{pb-diagram}
\usepackage{amsmath}
\usepackage{tikz-cd}
\usetikzlibrary{arrows.meta}
\tikzcdset{
every arrow/.append style={thick},
arrow style = tikz,            % TikZ 標準の矢印を使う
%diagrams = {>={Stealth[round]}} % 矢尻を Stealth(丸み付き)に
diagrams = {>={Latex[round]}}
}
\setlength{\dgARROWLENGTH}{1.4em}
\usepackage[left=1cm]{geometry}


\begin{document}
\(
\begin{tikzcd}[
nodes={font=\huge,inner sep=1pt},
row sep=0.3cm,
column sep=0.1cm,
column 16/.append style={nodes={align=left,anchor=west}}
]
n= & & 3 & & 5 & & 7 & & 9 & & 11 & & 13 & & 15 & {} \\
& {}  & {}  & ab  & {} & {} \\ 
k=20 & & & & \bullet \arrow[lu] & = & \bullet & = & \bullet & = & \bullet & = & & & & \{\beta_1^2\} \\
& & & AB & & b \\
k=21 & & & & \bullet \arrow[lu] & & \bullet \arrow[lu] & = & \bullet & & & & & & & \Delta(v(5))=P(B)  \\
& & & & & B \arrow[lu] & & & & a \arrow[lu] \\
k=22 & & & & & & & & & & (\bullet) & & & & & w=I(P(i))\ne0  \\
& b^2  & & & & & & & & & & (i) \arrow[lu] \\ 
k=23 & & \bullet \arrow[lu] & = & \bullet & = & \bullet & = & \bullet & = & \bullet & = & \bullet & = & & \{\alpha_1\beta_1^2\} \\
n= & & 3 & & 5 & & 7 & & 9 & & 11 & & 13 & & 15 & {} \\
& B^2 & & & & ab \\
k=24 & & \bullet \arrow[lu] & = & \bullet & & \bullet \arrow[lu] & & & & & & & & & \Delta(P(b))=P(AB) \\
& & & & & AB \arrow[lu] & & b \arrow[lu] \\
k=25 \\
& ab^2 & & & & & & B & & \\ 
k=26 & & \bullet \arrow[lu] & & & & & & \bullet \arrow[lu] & = & \bullet & = & \bullet & = & \bullet & = \{\beta_2\} \\
& AB^2 & & b^2 \arrow[lu] \\
k=27 & & \bullet \arrow[lu] \\
& & & B^2 \arrow[lu] \\
\end{tikzcd}
\)
\end{document}


